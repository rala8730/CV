%%%%%%%%%%%%%%%%%%%%%%%%%%%%%%%%%%%%%%%%%
% "ModernCV" CV and Cover Letter
% LaTeX Template
% Version 1.2 (25/3/16)
%
% This template has been downloaded from:
% http://www.LaTeXTemplates.com
%
% Original author:
% Xavier Danaux (xdanaux@gmail.com) with modifications by:
% Vel (vel@latextemplates.com)
%
% License:
% CC BY-NC-SA 3.0 (http://creativecommons.org/licenses/by-nc-sa/3.0/)
%
% Important note:
% This template requires the moderncv.cls and .sty files to be in the same 
% directory as this .tex file. These files provide the resume style and themes 
% used for structuring the document.
%
%%%%%%%%%%%%%%%%%%%%%%%%%%%%%%%%%%%%%%%%%

%----------------------------------------------------------------------------------------
%	PACKAGES AND OTHER DOCUMENT CONFIGURATIONS
%----------------------------------------------------------------------------------------
\documentclass[10pt,a4paper,sans]{moderncv} % Font sizes: 10, 11, or 12; paper sizes: a4paper, letterpaper, a5paper, legalpaper, executivepaper or landscape; font families: sans or roman
\usepackage{lmodern}
\usepackage{tasks}

%\usepackage{hyperref}
%\definecolor{darkblue}{rgb}{0.0,0.0,0.3}
%\hypersetup{colorlinks,breaklinks,linkcolor=darkblue,urlcolor=darkblue,anchorcolor=darkblue,citecolor=darkblue}
\moderncvstyle{banking} % CV theme - options include: 'casual' (default), 'classic', 'oldstyle' and 'banking'
\moderncvcolor{blue} % CV color - options include: 'blue' (default), 'orange', 'green', 'red', 'purple', 'grey' and 'black'

\usepackage{lipsum} % Used for inserting dummy 'Lorem ipsum' text into the template

\usepackage[scale=0.87]{geometry} % Reduce document margins
%\setlength{\hintscolumnwidth}{3cm} % Uncomment to change the width of the dates column
%\setlength{\makecvtitlenamewidth}{10cm} % For the 'classic' style, uncomment to adjust the width of the space allocated to your name

%----------------------------------------------------------------------------------------
%	NAME AND CONTACT INFORMATION SECTION
%----------------------------------------------------------------------------------------
\vspace{-10mm}
\firstname{Rasmi} % Your first name
\familyname{Lamichhane} % Your last name
% All information in this block is optional, comment out any lines you don't need
\title{R\'esum\'e}
%\address{417 Oak Ln}{Broomfield, CO 80020}
%\mobile{(720) 224 4940}
\extrainfo{\href{https://github.com/rala8730}{https://github.com/rala8730}}
\email{lamichhane.rasmi@gmail.com}
\email{rala8730@colorado.edu}
%\photo[70pt][0.4pt]{pictures/picture} % The first bracket is the picture height, the second is the thickness of the frame around the picture (0pt for no frame)

\begin{document}
\makecvtitle
%----------------------------------------------------------------------------------------
%	EDUCATION SECTION
%----------------------------------------------------------------------------------------
\vspace{-21mm}
\section{Education}
\cventry{\textnormal{2012-Current}}{\textnormal{Bachelor of Computer Science}}{The University of Colorado, Boulder}{}{}{}  % Arguments not required can be left empty
%----------------------------------------------------------------------------------------
%	WORK EXPERIENCE SECTION
%----------------------------------------------------------------------------------------
\vspace{-5mm}
\section{Experience}
\vspace{-1mm}
\cventry{\textnormal{Fall 2016-Present}}{\textnormal{Student Android Developer}}{\textsc{Laboratory For Playful Computation}}{\textnormal{Boulder, CO}}{}{\textnormal{}}
\vspace{-1mm}
\cventry{\textnormal{Fall 2015-Fall 2016}}{\textnormal{Student Assistant Digital Lab}}{\textsc{University of Colorado Boulder Libraries}}{\textnormal{Boulder, CO}}{}{\textnormal{Digitaling different kind of archival material and processing it.}}
%\hfill \break
\cventry{\textnormal{Summer 2015}}{\textnormal{Student Assistant LIT}}{\textsc{University of Colorado Boulder Libraries}}{\textnormal{Boulder, CO}}{}{\textnormal{Web help desk, Troubleshooting software and hardware problems, image computer, printers problems and etc.}}


%----------------------------------------------------------------------------------------
%	COMPUTER SKILLS SECTION
%----------------------------------------------------------------------------------------
\vspace{-5mm}
\section{Computer Skills}
\vspace{-2mm}
\cvitem{\textnormal{Language}}{C/C++, Python, Java, HTML/CSS, Mathematica, SQL, Bash Shell Scripting, Regex}{}{} 
\cvitem{\textnormal{Tools}}{Unit Testing, Pair Programming, Rest and Soap, Agile/Scrum methodology, Waterfall, Databases}
\cvitem{\textnormal{Extra}}{Github, Adobe Photoshop, Freehand, Microsoft Office}
\vspace{-5mm}
\section {Projects}
\vspace{-1mm}
\cventry{\href{https://youtu.be/TKgePysJxr0}{\textnormal{https://youtu.be/TKgePysJxr0}}{}}{}{\textsc{\href{https://git.io/v64Lp}{Sparki Pill Pusher}}}{\href{https://git.io/v64Lp}{\textnormal{https://git.io/v64Lp}}}{}{}
\begin{itemize}
\vspace{-6mm}
\item{Programmed a robot named Sparki. Sparki follows specified paths to fetch 1 of 5 bottles.} 
\item{It uses light sensors to make sure that it stays on the defined path, and uses its ultrasonic sensors to find the fastest path to the bottle at an optimal angle.} 
\item{Sparki uses its RFID scanner to scan the pill bottle, making sure that the unique tag  matches with the specified bottle number. If an incorrect bottle is approached, Sparki will beep, display a message that it is incorrect, and move back to the start to await more instructions.}
\end{itemize}

%\hfill \break
\cventry{\href{https://youtu.be/K5FWMMMd8d4}{\textnormal{https://youtu.be/K5FWMMMd8d4}}{}}{}{\textsc{\href{https://git.io/v68ax}{Python Data visualization}}}{\href{https://git.io/v68ax}{\textnormal{https://git.io/v68ax}}}{}{}
\begin{itemize}
\vspace{-6mm}
\item{Created a webpage using HTML, Python, and mySQL.}
\item{Displays an animated Chloropeth map to visualize Carbon Dioxide Emmissions per state in the US over the past two decades.}
\item{The states are color-coded according to the annual amount of emissions per state in million metric tons of carbon dioxide, and a cursor hover over a particular state will display a breakdown of that state's Carbon emissions per type in petroleum, coal, and gas.}
\end{itemize}
\cventry{}{}{\textsc{\href{https://git.io/viN7J}{Weather Data visualization}}}{\href{https://git.io/viN7J}{\textnormal{https://git.io/viN7J}}}{}{}
\begin{itemize}
\vspace{-6mm}
\item{Implemented a webpage that shows the weather across the United States. Used python programming language to grab the API so that it shows in the webpage. Used html for layouting and coloring the states and python for implementing different required functions.}
\end{itemize}
%\hfill \break
\cventry{}{}{\textsc{\href{https://git.io/viOZW}{Android Inventory App}}}{\href{https://git.io/viOZW}{\textnormal{https://git.io/viOZW}}}{}{}
\begin{itemize}
\vspace{-6mm}
\item{Implemented a inventory app using Android Studio with Java and XML. Displays image, price and quantity of the each item and calculates the overall amount of total items. Used various of textview, image view, buttons, linear and relative layout.}
\item{User can add each item by pressing the + button and remove the item by pressing - button. Textview shows the count of the items and price for each in the screen and finally submit shows the overall of cost of the transaction.}
\end{itemize}
\cventry{}{}{\textsc{\href{https://git.io/viNQe}{Rootfinding}}}{\href{https://git.io/viNQe}{\textnormal{https://git.io/viNQe}}}{}{}
\begin{itemize}
\vspace{-6mm}
\item{Implemented the Root-finding in python using newton's method, newton's method with line search and cubic methode. Used python programming with Dictionary and Tuple. Gives user a sence of difference between which function is diverging and conversing faster.}
\end{itemize}
\cventry{}{}{\textsc{\href{https://git.io/viNQI}{Battleship}}}{\href{https://git.io/viNQI}{\textnormal{https://git.io/viNQI}}}{}{}
\begin{itemize}
\vspace{-6mm}
\item{Implemented the battleship board game. The computer will hold the ships in the grid and the player will have to guess where those ships are. Used C++ classes, loops and different methodes. User can choose the size of the board location of the ship.}
\end{itemize}
\cventry{}{}{\textsc{\href{https://git.io/viNQ0}{Bag}}}{\href{https://git.io/viNQ0}{\textnormal{https://git.io/viNQ0}}}{}{}
\begin{itemize}
\vspace{-6mm}
\item{Implemented a bag of array and lists. Used C++ array, single and double linked list with classes, loops, pointers, and different methodes. The features are adding and removing the item in the pocket of magic, potion and good depending on user wants to add.}
\end{itemize}
\cventry{}{}{\textsc{\href{https://git.io/viNQA}{Stacks and queues}}}{\href{https://git.io/viNQA}{\textnormal{https://git.io/viNQA}}}{}{}
\begin{itemize}
\vspace{-6mm}
\item{Implemented stacks and queues of array, single linklist and double linked list. Used C++ array, single and double linked list with classes, loops, pointers, and different methodes. Features are adding and removing and moving element depend on users desire.}
\end{itemize}
%----------------------------------------------------------------------------------------
%	INTERESTS SECTION
%----------------------------------------------------------------------------------------
\vspace{-5mm}
\section{Extra Curricular}
ALP(Applied Leadership Program),CLP(Core Leadership Program),Graphic Designing,Westminister Public Library(Volunteer in the computer class)2014, CUWIC(Women in Computing)


%----------------------------------------------------------------------------------------
\end{document}

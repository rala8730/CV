%%%%%%%%%%%%%%%%%%%%%%%%%%%%%%%%%%%%%%%%%
% "ModernCV" CV and Cover Letter
% LaTeX Template
% Version 1.2 (25/3/16)
%
% This template has been downloaded from:
% http://www.LaTeXTemplates.com
%
% Original author:
% Xavier Danaux (xdanaux@gmail.com) with modifications by:
% Vel (vel@latextemplates.com)
%
% License:
% CC BY-NC-SA 3.0 (http://creativecommons.org/licenses/by-nc-sa/3.0/)
%
% Important note:
% This template requires the moderncv.cls and .sty files to be in the same 
% directory as this .tex file. These files provide the resume style and themes 
% used for structuring the document.
%
%%%%%%%%%%%%%%%%%%%%%%%%%%%%%%%%%%%%%%%%%

%----------------------------------------------------------------------------------------
%	PACKAGES AND OTHER DOCUMENT CONFIGURATIONS
%----------------------------------------------------------------------------------------
\documentclass[10pt,a4paper,sans]{moderncv} % Font sizes: 10, 11, or 12; paper sizes: a4paper, letterpaper, a5paper, legalpaper, executivepaper or landscape; font families: sans or roman
\usepackage{lmodern}

%\usepackage{hyperref}
%\definecolor{darkblue}{rgb}{0.0,0.0,0.3}
%\hypersetup{colorlinks,breaklinks,linkcolor=darkblue,urlcolor=darkblue,anchorcolor=darkblue,citecolor=darkblue}
\moderncvstyle{banking} % CV theme - options include: 'casual' (default), 'classic', 'oldstyle' and 'banking'
\moderncvcolor{blue} % CV color - options include: 'blue' (default), 'orange', 'green', 'red', 'purple', 'grey' and 'black'

\usepackage{lipsum} % Used for inserting dummy 'Lorem ipsum' text into the template

\usepackage[scale=0.87]{geometry} % Reduce document margins
%\setlength{\hintscolumnwidth}{3cm} % Uncomment to change the width of the dates column
%\setlength{\makecvtitlenamewidth}{10cm} % For the 'classic' style, uncomment to adjust the width of the space allocated to your name

%----------------------------------------------------------------------------------------
%	NAME AND CONTACT INFORMATION SECTION
%----------------------------------------------------------------------------------------

\firstname{Rasmi} % Your first name
\familyname{Lamichhane} % Your last name
% All information in this block is optional, comment out any lines you don't need
\title{R\'esum\'e}
%\address{417 Oak Ln}{Broomfield, CO 80020}
%\mobile{(720) 224 4940}
\extrainfo{\href{https://github.com/rala8730}{https://github.com/rala8730}}
\email{lamichhane.rasmi@gmail.com}
\email{rala8730@colorado.edu}
%\photo[70pt][0.4pt]{pictures/picture} % The first bracket is the picture height, the second is the thickness of the frame around the picture (0pt for no frame)

\begin{document}
\makecvtitle
%----------------------------------------------------------------------------------------
%	EDUCATION SECTION
%----------------------------------------------------------------------------------------

\section{Education}
\cventry{\textnormal{2012-Current}}{\textnormal{Bachelor of Computer Science}}{The University of Colorado, Boulder}{}{}{}  % Arguments not required can be left empty
%----------------------------------------------------------------------------------------
%	WORK EXPERIENCE SECTION
%----------------------------------------------------------------------------------------
\section{Experience}
\cventry{\textnormal{Fall 2015-Present}}{\textnormal{Student Assistant Digital Lab}}{\textsc{University of Colorado Boulder Libraries}}{\textnormal{Boulder, CO}}{}{\textnormal{Digitaling different kind of archival material and processing it.}}
%------------------------------------------------
\hfill \break
\cventry{\textnormal{Summer 2015}}{\textnormal{Student Assistant LIT}}{\textsc{University of Colorado Boulder Libraries}}{\textnormal{Boulder, CO}}{}{\textnormal{Web help desk, Troubleshooting software and hardware problems, image computer, printers problems and etc.}}
%----------------------------------------------------------------------------------------
%	COMPUTER SKILLS SECTION
%----------------------------------------------------------------------------------------
\section{Computer Skills}
\cvitem{\textnormal{Language}}{C/C++, Python, Java, HTML/CSS, Mathematica, SQL, Bash Shell Scripting, Regex}{}{} 
\cvitem{\textnormal{Tools}}{Unit Testing, Pair Programming, Rest and Soap, Agile/Scrum methodology, Waterfall, Databases}
\cvitem{\textnormal{Extra}}{Github, Adobe Photoshop, Freehand, Microsoft Office}

\section {Projects}
\cventry{\href{https://youtu.be/TKgePysJxr0}{\textnormal{https://youtu.be/TKgePysJxr0}}{}}{}{\textsc{\href{https://git.io/v64Lp}{Sparki Pill Pusher}}}{\href{https://git.io/v64Lp}{\textnormal{https://git.io/v64Lp}}}{}{}
\begin{itemize}
\item{Sparki follows specified paths to fetch 1 of 5 bottles.} 
\item{It uses light sensors to make sure that it stays on the defined path, and uses its ultrasonic sensors to find the fastest path to the bottle at an optimal angle.} 
\item{Sparki uses its RFID scanner to scan the pill bottle, making sure that the unique tag  matches with the specified bottle number. If an incorrect bottle is approached, Sparki will beep, display a message that it is incorrect, and move back to the start to await more instructions.}
\end{itemize}

\hfill \break
\cventry{\href{https://youtu.be/K5FWMMMd8d4}{\textnormal{https://youtu.be/K5FWMMMd8d4}}{}}{}{\textsc{\href{https://git.io/v68ax}{Python Data visualization}}}{\href{https://git.io/v68ax}{\textnormal{https://git.io/v68ax}}}{}{}
\begin{itemize}
\item{ Created a webpage using HTML, Python, and mySQL, which displays an animated Chloropeth map to visualize Carbon Dioxide Emmissions per state in the United States over the past two decades.}
\item{The states are color-coded according to the annual amount of emissions per state in million metric tons of carbon dioxide, and a cursor hover over a particular state will display a breakdown of that state's Carbon emissions per type in petroleum, coal, and gas.}
\end{itemize}
%----------------------------------------------------------------------------------------
%	INTERESTS SECTION
%----------------------------------------------------------------------------------------

\section{Extra Curricular}
\begin{itemize}
\item{}{ALP(Applied Leadership Program)}
\item{}{CLP(Core Leadership Program)}
\item{}{Graphic Designing}
\item{}{Westminister Public Library(Volunteer in the computer class){2014}}
\end{itemize}

\section{Groups}
\begin{itemize}
\item {}{HACK CU,CUWIC(Women in Computing)}
\end{itemize}

%----------------------------------------------------------------------------------------
\end{document}